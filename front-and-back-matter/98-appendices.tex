% Options for packages loaded elsewhere
\PassOptionsToPackage{unicode}{hyperref}
\PassOptionsToPackage{hyphens}{url}
%
\documentclass[
]{article}
\usepackage{lmodern}
\usepackage{amssymb,amsmath}
\usepackage{ifxetex,ifluatex}
\ifnum 0\ifxetex 1\fi\ifluatex 1\fi=0 % if pdftex
  \usepackage[T1]{fontenc}
  \usepackage[utf8]{inputenc}
  \usepackage{textcomp} % provide euro and other symbols
\else % if luatex or xetex
  \usepackage{unicode-math}
  \defaultfontfeatures{Scale=MatchLowercase}
  \defaultfontfeatures[\rmfamily]{Ligatures=TeX,Scale=1}
\fi
% Use upquote if available, for straight quotes in verbatim environments
\IfFileExists{upquote.sty}{\usepackage{upquote}}{}
\IfFileExists{microtype.sty}{% use microtype if available
  \usepackage[]{microtype}
  \UseMicrotypeSet[protrusion]{basicmath} % disable protrusion for tt fonts
}{}
\makeatletter
\@ifundefined{KOMAClassName}{% if non-KOMA class
  \IfFileExists{parskip.sty}{%
    \usepackage{parskip}
  }{% else
    \setlength{\parindent}{0pt}
    \setlength{\parskip}{6pt plus 2pt minus 1pt}}
}{% if KOMA class
  \KOMAoptions{parskip=half}}
\makeatother
\usepackage{xcolor}
\IfFileExists{xurl.sty}{\usepackage{xurl}}{} % add URL line breaks if available
\IfFileExists{bookmark.sty}{\usepackage{bookmark}}{\usepackage{hyperref}}
\hypersetup{
  hidelinks,
  pdfcreator={LaTeX via pandoc}}
\urlstyle{same} % disable monospaced font for URLs
\usepackage[margin=1in]{geometry}
\usepackage{graphicx,grffile}
\makeatletter
\def\maxwidth{\ifdim\Gin@nat@width>\linewidth\linewidth\else\Gin@nat@width\fi}
\def\maxheight{\ifdim\Gin@nat@height>\textheight\textheight\else\Gin@nat@height\fi}
\makeatother
% Scale images if necessary, so that they will not overflow the page
% margins by default, and it is still possible to overwrite the defaults
% using explicit options in \includegraphics[width, height, ...]{}
\setkeys{Gin}{width=\maxwidth,height=\maxheight,keepaspectratio}
% Set default figure placement to htbp
\makeatletter
\def\fps@figure{htbp}
\makeatother
\setlength{\emergencystretch}{3em} % prevent overfull lines
\providecommand{\tightlist}{%
  \setlength{\itemsep}{0pt}\setlength{\parskip}{0pt}}
\setcounter{secnumdepth}{-\maxdimen} % remove section numbering

\author{}
\date{\vspace{-2.5em}}

\begin{document}

\startappendices

\hypertarget{chap:appendix_a}{%
\section{Backpropagation with Binary Cross
Entropy}\label{chap:appendix_a}}

Let's consider a simple binary classification task, it is common to use
a network with a single logistic output with the binary cross-entropy
loss function and for the sake of simplicity let's assume that there is
only one hidden layer. \[
\begin{aligned}
BCE=-\sum_{i=1}^{n o u t}\left(y_i \log \left(\hat{y}_i \right)+\left(1-y_i\right) \log \left(1-\hat{y}_i\right)\right)
\end{aligned}
\]

where \(y\) is the ground truth and \(\hat{y}\) is the output of the
network. After having the loss function let's continue with the forward
pass.

\[
\begin{aligned} 
a_{k} &= h_{k-1} w_{k} + b_k \\
h_k &= f(a_{k})
\end{aligned}
\]

where, \(w_k\) is the weight, \(b_{k}\) is the bias term, \(h_k\) is the
output of the layer (which means that \(h_0 = X\) and \(h_2 = \hat{y}\))
and f is the non linear function. Please note that for last layer
logistic function is used whereas for hidden layer reLU is used as non
linear functions.\\
We can compute the derivative of the weights by using the chain rule.

\[
\begin{aligned} 
\frac{\partial BCE}{\partial w_{2}}=\frac{\partial BCE}{\partial \hat{y}} \frac{\partial \hat{y}}{\partial a_{2}} \frac{\partial a_{2}}{\partial w_{2}}
\end{aligned}
\]

Computing each factor in the term, we have: \[
\begin{aligned}
\frac{\partial BCE}{\partial \hat{y}} &=\frac{-y}{\hat{y}}+\frac{1-y}{1-\hat{y}} \\
&=\frac{\hat{y}-y}{\hat{y}\left(1-\hat{y}\right)} \\
\frac{\partial \hat{y}}{\partial a_{2}} &=\hat{y}\left(1-\hat{y}\right) \\
\frac{\partial a_{2}}{\partial w_{2}} &=h_{1}
\end{aligned}
\] which gives us: \[
\frac{\partial BCE}{\partial w_{2}}=h_{1}^T\left(\hat{y}-y\right)
\] Derivative of the \(w_1\) with respect to loss function can be
calculated as the following:

\[
\begin{aligned} 
\frac{\partial BCE}{\partial w_{1}}=\frac{\partial BCE}{\partial h_1} \frac{\partial h_1}{\partial a_{1}} \frac{\partial a_{1}}{\partial w_{1}}
\end{aligned}
\] Compute each factor in the term again, we have:

\[
\begin{aligned}
\frac{\partial BCE}{\partial h_1} &= \frac{\partial BCE}{\partial \hat{y}} \frac{\partial \hat{y}}{\partial a_{2}} \frac{\partial a_{2}}{\partial h_{1}}  \\
&= \left(\hat{y}-y\right) w_{2} \\
\frac{\partial h_1}{\partial a_{1}} &=f'(a_1) \\
\frac{\partial a_{1}}{\partial h_{1}} &=X
\end{aligned}
\] which gives us: \[
\begin{aligned}
\frac{\partial BCE}{\partial w_{1}}= \left(X\right)^T\left(\hat{y}-y\right)\left(w_{2}^T\right) \odot f'(a_1)
\end{aligned}
\] where \(\odot\) is element-wise multiplication. Similarly, bias terms
can be calculated by following:

\[
\begin{aligned} 
\frac{\partial BCE}{\partial b_{2}}&=\frac{\partial BCE}{\partial \hat{y}} \frac{\partial \hat{y}}{\partial a_{2}} \frac{\partial a_{2}}{\partial b_{2}} \\
&= \left(\hat{y}-y\right)
\end{aligned}
\]

\[
\begin{aligned} 
\frac{\partial BCE}{\partial b_{1}}&=\frac{\partial BCE}{\partial h_1} \frac{\partial h_1}{\partial a_{1}} \frac{\partial a_{1}}{\partial b_{1}} \\
&= \left(\hat{y}-y\right)\left(w_{2}^T\right) \odot f'(a_1)
\end{aligned}
\] After having all these results, we can simply update the parameters
(weights and biases) by using gradient descent and its variants as
follows:

\[
\begin{aligned} 
\text{parameter} &= \text{parameter} - \text{step size} \times \frac{\partial BCE}{\partial \text(parameter)}  \\
\end{aligned}
\]

\hypertarget{reproducibility}{%
\section{Reproducibility}\label{reproducibility}}

\end{document}
